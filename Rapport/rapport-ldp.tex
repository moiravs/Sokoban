
\documentclass[utf8]{article}
\usepackage[utf8]{inputenc}

\usepackage[parfill]{parskip}
\usepackage{booktabs}
\usepackage{amsmath}
\usepackage{amssymb}
\usepackage{amsfonts}
\usepackage{graphicx}
\usepackage{float}
\usepackage{listingsutf8}
\usepackage{listings}
\usepackage{graphicx}
\usepackage{fullpage}
\usepackage{lipsum}

\nocite{*}
\setlength\parindent{24pt}

\begin{document}
\begin{titlepage}


\author{Andrius Ezerskis \& Moïra Vanderslagmolen}
\title{Projet de Langage de Programmation: Sokoban}

\maketitle
\end{titlepage}
\begin{large}

\section{Introduction}
\indent
\par
Nous avons séparé ce rapport en plusieurs parties. Tout d'abord, nous présentons
les tâches accomplies. Ensuite, nous verrons les différentes classes
implémentées et nous les expliquerons. Par après, nous parlerons de la logique
du jeu, nous décrirons le déroulement d'un début de jeu, d'une fin de jeu, de
l'enregistrement des steps et enfin des widgets. Pour finir, nous discuterons du
modèle MVC dans notre projet.

\par

\section{Tâches Accomplies}

\par
\indent

Nous avons accompli 10 tâches.
Nous avons implémenté :

\begin{itemize}
    \item Les boîtes de couleur
    \item Les cases de téléportation
    \item Les boîtes légères
    \item Le compteur de steps
    \item Le meilleur score de steps
    \item L'écran d'accueil
    \item Les niveaux et sélection de niveaux 
    \item La limite de steps
    \item Le déplacement automatique à la souris
    \item La détection d'échec
  \end{itemize}

\par


\section{Classes}
\indent
\par
Nous avons séparé nos classes en trois fichiers distincts : Model, Controller et View.
\par
\subsection{Classes du Modèle}

\subsubsection{BoardModel}
\indent
\par
Nous avons le BoardModel, qui s'occupe de gérer la logique du plateau. Elle fait
bouger le personnage, compte le nombre de steps, détermine si la partie est
terminée ou si elle ne peut plus être résolue. Elle peut aussi téléporter le
personnage d'une case de téléportation à une autre et charger les niveaux en
mémoire.
\par
Les niveaux sont affichés comme ceci : la première ligne indique la limite de
pas. Elle affiche un l et un nombre ensuite. La deuxième ligne indique le
meilleur score, le fichier écrit un m suivi du meilleur score.
\par

\subsubsection{LogicCell}
\indent
\par
Cette classe représente une cellule ou une case. Elle a une position, peut avoir
un joueur ou alors une boîte. Elle a aussi une couleur et un type. Elle peut
avoir 4 types différents (mur, vide, téléportation ou position finale d'une
boîte). Si le type est la position finale d'une boîte, alors elle est "complète"
que si la couleur de LogicCell correspond à celle de la boîte. La cellule est
bloquée si son type est un mur ou si la boîte sur la case est bloquée.
\par
\subsubsection{Téléportation}
\indent
\par
La classe téléportation est un peu plus complexe. En effet, elle prend deux
LogicCell en paramètre. 
\par
\subsection{Classes de la Vue}
\subsubsection{CellDisplay}
\indent
\par
Cette classe s'occupe de dessiner chaque cellule en fonction de son type
\par
\subsubsection{DisplayBoard}
\indent
\par
Le DisplayBoard va itérer à travers les LogicCell et créer des instances de
CellDisplay. Enfin, lorsque l'utilisateur clique sur le
DisplayBoard, il va demander à chaque Cell si elle a été cliquée et renverra le
résultat au MainWindow.
\par
\subsubsection{HelpWindow}
\indent
\par
Cette fenêtre s'ouvre quand l'utilisateur appuie sur le bouton Help. Elle
affiche toutes les commandes disponibles
\par
\subsubsection{StartWindow}
\indent
\par
Cette fenêtre permet d'afficher un écran d'accueil et affiche le nom des
auteurs. Elle se lance en premier et disparaît après 10 secondes.
\par

\subsubsection{PopUp}
\indent
\par
Cette fenêtre s'ouvre lorsque l'utilisateur clique sur le bouton Levels. Elle
affiche tous les niveaux disponibles 

\subsection{Classes du Controlleur}
\subsubsection{MainWindow}
\indent
\par
Cette classe sert de controlleur. Il va aussi dessiner les limite de steps, le
compteur de pas et le nombre de pas minimum sur la fenêtre. Cette classe
gère les commandes entrées par l'utilisateur. Il s'occupe aussi des clicks.
Il affiche aussi la fenêtre principale du jeu.
\par
\section{Logique du jeu}
\subsection{Démarrer le jeu}
\indent
\par

Lorsque nous démarrons le jeu, nous donnons un fichier au BoardModel, qui va
créer un vecteur de LogicCell et enregistrer la limite de pas ainsi que le
meilleur score dans une variable.
\par
Ensuite, nous allons créer une instance de popUp et de HelpWindow, et nous les
passons ensuite en paramètre à MainWindow. Le MainWindow va dessiner tous les
widgets du jeu(bouton d'aide, changer de niveau, recommencer le niveau et reset
le meilleur score). MainWindow va aussi dessiner le nombre de pas, la limite de
pas ainsi que le meilleur score. Le DisplayBoard va dessiner toutes les
CellDisplay une par une grâce au vecteur de LogicCell créé dans BoardModel.


\par
\subsection{Jouer}
\indent
\par
Le mainWindow s'occupe de gérer chaque événement. Par exemple, lorsque
l'utilisateur appuie sur la flèche de droite et que la partie n'est pas
terminée, le mainWindow va appeler la classe BoardModel, va mettre à jour le
plateau de jeu et va redessiner les pas. Si l'utilisateur clique sur le board,
MainWindow va demander à DisplayBoard quelle est la position de la case cliquée.
Celui-ci renverra un tuple et MainWindow enverra ce tuple à BoardModel.
BoardModel va mettre à jour le vecteur de logicCell et la position du player. Il
vérifie ensuite si la partie est terminée ou pas.
\par
Si la partie est terminée, MainWindow affichera un message indiquant à
l'utilisateur qu'il a gagné ou perdu.
\par

\subsection{Widgets}
\subsubsection{Help}
\indent
\par
Si l'utilisateur clique sur le bouton Help, une nouvelle fenêtre va s'ouvrir. En
réalité, le code va juste montrer la fenêtre grâce à la fonction show.
\par
\subsubsection{Levels}
\indent
\par
Si l'utilisateur appuie sur le bouton changeLevel, une nouvelle fenêtre va être
affichée. Tant que la fenêtre est affichée, nous appelons la fonction Fl::wait.
La fenêtre est modale, c'est à dire que l'utilisateur ne peut plus faire aucune
autre action dans le jeu tant qu'il n'a pas cliqué sur le bouton pour choisir le
niveau. Lorsque l'utilisateur a cliqué sur un bouton, le callback de ce bouton
va fermer la fenêtre. Tout d'abord, MainWindow va demander à BoardModel
d'enregistrer le meilleur score et ensuite, il va récupérer l'information de
PopUp et enverra l'information à BoardModel.
\par

\subsubsection{ResetMinPas}
\indent
\par
Si l'utilisateur appuie sur le bouton pour remettre à 0 son meilleur score,
MainWindow va appeler boardModel et mettre le nombre de pas à 0. BoardModel va
écrire dans le fichier que le meilleur score est égal à 0. Pour écrire dans le
fichier, il cherche m + l'ancien meilleur score. Si il le trouve, il le remplace
par m + le nouveau meilleur score.
\par

\subsubsection{ResetLevel}
Pour recommencer un niveau, nous avons décidé de recharger le niveau à partir du
fichier. Il agit donc comme un changement de niveau.
\subsection{Quitter le jeu}
\indent
\par
Lorsque nous quittons le jeu, le callback s'occupant de fermer la fenêtre va
enregistrer le meilleur score de l'utilisateur.
\par

\section{Modèle-Vue-Controlleur}
\indent
\par
Nous avons implémenté le modèle MVC uniquement pour le board. Premièrement, nous
avons d'abord essayé d'implémenté tout le jeu en modèle MVC, mais FLTK n'est pas
adapté au modèle MVC. En effet, les callback des widgets sont premièrement
statiques, nous étions donc obligés d'appeller notre Controller dans le
MainWindow, ce qui n'avait pas beaucoup de sens. 
\par
Nous avons donc décidé que la
fenêtre principale serait le Controlleur, vu qu'il gère les commandes entrées
par l'utilisateur et met à jour le boardModel  et le display en fonction.
\par
\section{Conclusion}
\indent
\par
En Conclusion,
\par

\end{large}

\end{document}
