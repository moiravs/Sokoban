
\documentclass[utf8]{article}
\usepackage[utf8]{inputenc}

\usepackage[parfill]{parskip}
\usepackage{booktabs}
\usepackage{amsmath}
\usepackage{amssymb}
\usepackage{amsfonts}
\usepackage{graphicx}
\usepackage{float}
\usepackage{listingsutf8}
\usepackage{listings}
\usepackage{graphicx}
\usepackage{fullpage}
\usepackage{lipsum}

\nocite{*}
\setlength\parindent{24pt}

\begin{document}
\begin{titlepage}


\author{Andrius Ezerskis \& Moïra Vanderslagmolen}
\title{Projet de Langage de Programmation: Sokoban}

\maketitle
\end{titlepage}
\begin{large}

\section{Introduction}
\indent
\par
Nous avons séparé ce rapport en plusieurs parties. Tout d'abord, nous présentons
les tâches accomplies. Ensuite, nous verrons les différentes classes
implémentées et nous les expliquerons. Par après, nous parlerons de la logique
du jeu, nous décrirons le déroulement d'un début de jeu, d'une fin de jeu, de
l'enregistrement des steps et enfin des widgets. Pour finir, nous discuterons du
modèle MVC dans notre projet.

\par

\section{Tâches Accomplies}

\par
\indent

Nous avons accompli 10 tâches.
Nous avons implémenté :

\begin{itemize}
    \item Les boîtes de couleur
    \item Les cases de téléportation
    \item Les boîtes légères
    \item Le compteur de steps
    \item Le meilleur score de steps
    \item L'écran d'accueil
    \item Les niveaux et sélection de niveaux 
    \item La limite de steps
    \item Le déplacement automatique à la souris
    \item La détection d'échec
  \end{itemize}

\par


\section{Classes}
\indent
\par
Nous avons séparé nos classes en trois fichiers distincts : Model, Controller et Vue.
\par
\subsection{Classes du Modèle}

\subsubsection{BoardModel}
\indent
\par
BoardModel est la classe qui s'occupe de gérer la logique du plateau. Elle fait
bouger le personnage, compte le nombre de pas, détermine si la partie est
terminée ou si elle ne peut plus être résolue. Elle peut aussi téléporter le
personnage d'une case de téléportation à une autre et charger les niveaux en
mémoire et s'occupe de sauvegarder le meilleur score en mémoire.
\par
\indent
\par
Nous avons mis les niveaux dans des fichiers, nous avons donc aussi implémenté
une fonction qui permet de vérifier si le plateau de jeu est fonctionnel. Nous
vérifions si le plateau de jeu a bien un joueur, si les téléportations ont bien une
case arrivée et si le plateau de jeu a des lignes de même taille.
\par
Les niveaux sont affichés comme ceci : la première ligne indique la limite de
pas. Elle affiche un l et un nombre ensuite. La deuxième ligne indique le
meilleur score, le fichier affiche un m suivi du meilleur score.
\par

\subsubsection{LogicCell}
\indent
\par
Nous avons implémenté deux structures : Player et Box, que nous avons placé dans
le fichier de LogicCell. Player contient seulement deux attributs, x et y, qui
représentent la ligne et la colonne dans laquelle est placé le player. Box
contient trois attributs, la couleur de la boîte, un booléen indiquant si elle
est légère et un autre booléen indiquant si la boîte est bloquée.
\par
Cette classe représente une cellule ou une case. Elle a une position, et
contient un joueur ou une boîte. La case a aussi une couleur et un type. Elle
peut avoir 4 types différents (mur, vide, téléportation, position finale d'une
boîte). 
\par
Nous avons écrit une méthode qui permet de vérifier si la case est
"complète". En effet, si le type de la case est la position finale d'une boîte,
alors elle est complète que si elle possède une boîte et si cette boîte a la
même couleur que la cellule qui la contient. Elle est aussi complète lorsque le
type n'est pas la position finale d'une boîte. Cette fonction est utile
lorsque nous voulons vérifier si le joueur a terminé son niveau, nous itérons alors
dans le vecteur contenant toutes les cases et nous vérifions si chacune des
cases est complète. La cellule est bloquée si son type est un mur ou si la boîte
sur la case est bloquée.
\par
\subsubsection{Téléportation}
\indent
\par
La classe téléportation prend deux LogicCell en paramètre, la case de départ et
la case d'arrivée. Elle contient aussi une méthode qui prend un tuple d'int (la
position du joueur), et qui renvoie un tuple de la position de la case d'arrivée
de la téléportation si le joueur est sur une case de téléportation ou un tuple
de -1 sinon.
\par
\subsection{Classes de la Vue}
\subsubsection{CellDisplay}
\indent
\par
Cette classe s'occupe de dessiner chaque cellule en fonction de son type. Elle
contient aussi une méthode qui renvoie la position de la cellule si elle
contient la position de la souris.
\par
\subsubsection{DisplayBoard}
\indent
\par
Le DisplayBoard va itérer à travers le vecteur de LogicCell et créer des
instances de CellDisplay. Enfin, lorsque l'utilisateur clique sur le
DisplayBoard, il va demander à chaque instance de CellDisplay si elle
contient la position de la souris et renverra le résultat au MainWindow.
\par
\subsubsection{HelpWindow}
\indent
\par
Cette fenêtre s'ouvre quand l'utilisateur appuie sur le bouton Help. Elle
affiche toutes les commandes disponibles. Cette fenêtre permet aussi d'afficher
les niveaux disponibles, lorsque l'utilisateur appuie sur le bouton Levels.
\par
\subsubsection{StartWindow}
\indent
\par
Cette fenêtre permet d'afficher un écran d'accueil et affiche le nom des
auteurs. C'est la première fenêtre qui s'affiche et disparaît après 10 secondes.
\par

\subsection{Classes du contrôleur}
\subsubsection{MainWindow}
\indent
\par
Cette classe sert de contrôleur. En effet, elle s'occupe de faire le lien entre
le modèle et la vue. Elle va aussi dessiner les limites de pas, le compteur de
pas et le meilleur score. Cette classe gère aussi les commandes entrées par
l'utilisateur et s'occupe aussi des callback de chaque bouton. Chaque callback
change l'information dans le modèle et met à jour le display en fonction. Il
affiche aussi la fenêtre principale du jeu.
\par
\section{Logique du jeu}
\subsection{Démarrer le jeu}
\indent
\par

Lorsque nous démarrons le jeu, nous donnons un fichier au BoardModel, qui va
créer un vecteur de LogicCell, vérifier si le niveau est fonctionnel et
enregistrer la limite de pas ainsi que le meilleur score dans une variable.
\par
Ensuite, nous allons créer une instance de HelpWindow, et nous les passons
ensuite en paramètre à MainWindow. MainWindow va dessiner tous les widgets du
jeu(bouton d'aide, changer de niveau, recommencer le niveau et remettre à 0 le
meilleur score). MainWindow va aussi dessiner le nombre de pas, la limite de pas
ainsi que le meilleur score. Le DisplayBoard, créé dans mainWindow, va dessiner
toutes les CellDisplay une par une grâce au vecteur de LogicCell créé dans
BoardModel. 


\par
\subsection{Jouer}
\indent
\par
La classe MainWindow s'occupe de gérer chaque événement. Par exemple, lorsque
l'utilisateur appuie sur la flèche de droite et que la partie n'est pas
terminée, la classe MainWindow va tout d'abord appeler la classe BoardModel,
pour lui indiquer que le joueur veut aller à droite. Le BoardModel va mettre à
jour son vecteur de LogicCell et le nombre de pas. MainWindow va demander à
DisplayBoard de se mettre à jour, et va donc recréer un nouveau vecteur de
CellDisplay grâce au vecteur de LogicCell de BoardModel. MainWindow va aussi
redessiner les pas. 
\par
Si l'utilisateur clique sur le plateau de jeu, MainWindow va demander à
DisplayBoard quelle est la position de la case cliquée. Celui-ci va demander à
chaque CellDisplay si la position de la souris est dans cette CellDisplay. Si la
CellDisplay renvoie true, il envoie le tuple de cette Cell à MainWindow, qui
enverra ce tuple à BoardModel. BoardModel va mettre à jour le vecteur de
logicCell et la position du joueur. MainWindow vérifie ensuite si la partie est
terminée ou pas.
\par
Si la partie est terminée, les évènements liés au plateau de jeu ne sont plus
reconnus par MainWindow et celui-ci affichera un message indiquant à
l'utilisateur s'il a gagné ou perdu.
\par

\subsection{Widgets}
\subsubsection{Help}
\indent
\par
Si l'utilisateur clique sur le bouton Help, une nouvelle fenêtre va s'ouvrir.
Nous mettons l'attribut de help à true, pour indiquer à helpWindow d'afficher
l'aide. Ensuite, nous appelons la fonction show de Fl\_Window sur l'instance de
HelpWindow.
\par
\subsubsection{Levels}
\indent
\par
Si l'utilisateur appuie sur le bouton changeLevel, une nouvelle fenêtre va être
affichée grâce à la fonction show de FLTK. Tant que la fenêtre est affichée (la
fonction shown de Fl\_Window), nous appelons la fonction Fl::wait. La fenêtre est
modale, c'est à dire que l'utilisateur ne peut plus faire aucune autre action
dans une autre fenêtre tant qu'il n'a pas cliqué sur le bouton pour choisir le
niveau. Lorsque l'utilisateur a cliqué sur un des niveaux, le callback de ce
bouton va fermer la fenêtre. Ensuite, MainWindow va demander à BoardModel
d'enregistrer le meilleur score va récupérer l'information de HelpWindow et
enverra l'information à BoardModel, qui va recréer un nouveau board. Enfin,
MainWindow va demander à DisplayBoard de se mettre à jour.

\par

\subsubsection{ResetMinPas}
\indent
\par
Si l'utilisateur appuie sur le bouton pour remettre à 0 son meilleur score,
MainWindow va appeler boardModel et mettre le nombre de pas à 0. BoardModel va
écrire dans le fichier que le meilleur score est égal à 0. Pour écrire dans le
fichier, il cherche m + l'ancien meilleur score. Lorsqu'il le trouve, il le
remplace par m + le nouveau meilleur score. MainWindow va appeler sa fonction
draw pour redessiner le meilleur score.
\par

\subsubsection{ResetLevel}
Pour recommencer un niveau, nous avons décidé de recharger le niveau à partir du
fichier. Il agit donc comme un changement de niveau.
\subsection{Quitter le jeu}
\indent
\par
Lorsque nous quittons le jeu, le callback s'occupant de fermer la fenêtre va
enregistrer le meilleur score de l'utilisateur.
\par

\section{Modèle-Vue-contrôler}
\indent
\par
Nous avons implémenté le modèle MVC uniquement pour le board. Premièrement, nous
avons d'abord essayé d'implémenter tout le jeu en modèle MVC, mais FLTK n'est
pas adapté au modèle MVC. En effet, les callback des widgets sont statiques et
nous étions donc obligés d'appeler notre contrôleur dans le MainWindow. Nous
avons donc décidé que la fenêtre principale serait le contrôleur du board vu
qu'il gère les commandes entrées par l'utilisateur et met à jour le boardModel
et le display en fonction et il s'occupe aussi de l'affichage des boutons et de
leur callback.
\par
Nous avons aussi décidé par souci de performance d'implémenter la vue dans le
contrôleur. En effet, nous avions implémenté une fonction qui rafraîchit le
board 60 fois par seconde, et il n'était donc pas nécessaire d'avoir cette
classe dans le contrôleur. Seulement, cela ralentissait beaucoup notre programme
et le jeu clignotait. Nous avons donc mis DisplayBoard dans MainWindow. Lorsque
nous avons une commande entrée par l'utilisateur, nous mettons à jour le plateau
de jeu (le modèle) et nous appelons la fonction update de display pour
lui dire de se mettre à jour. 
\par
\section{Conclusion}
\indent
\par
En conclusion, nous avons implémenté les 10 tâches requises, nous avons écrit
plusieurs classes, séparés dans des fichiers différents et dans trois dossiers
différents : Modèle, Vue et Contrôleur. Nous avons essayé de respecter le modèle
MVC, mais par souci de performance, nous avons décidé de mettre la vue dans le
contrôleur, pour qu'il puisse être notifié des changements du boardModel.
\par

\end{large}

\end{document}
