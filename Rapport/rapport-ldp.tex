
\documentclass[utf8]{article}
\usepackage[utf8]{inputenc}

\usepackage[parfill]{parskip}
\usepackage{booktabs}
\usepackage{amsmath}
\usepackage{amssymb}
\usepackage{amsfonts}
\usepackage{graphicx}
\usepackage{float}
\usepackage{listingsutf8}
\usepackage{listings}
\usepackage{graphicx}
\usepackage{fullpage}
\usepackage{lipsum}

\nocite{*}
\setlength\parindent{24pt}

\begin{document}
\begin{titlepage}


\author{Andrius Ezerskis \& Moïra Vanderslagmolen}
\title{Projet de Language de Programmation Sokoban}

\maketitle
\end{titlepage}
\begin{large}

\section{Introduction}
\par
\indent
\par

\section{Tâches Accomplies}
\par
\indent
Nous avons accompli 10 tâches.
Nous avons tout d'abord implémenté les boîtes de couleur.
Cases de téléportation
Boites légères
Compteur de par
Meilleur score de pas
Ecran daccueil
Niveaux et sélection de Niveaux 
limite de pas
déplacement automatique à la souris et détection d'échec


\par


\section{Classes}
\par
\indent

\par
\subsection{Classes du Modèle}

\subsubsection{BoardModel}
\par
\indent
Nous avons le BoardModel, qui s'occupe de gérer la logique du plateau. Elle fait bouger le personnage,
compte le nombre de pas, 
Ensuite
\par
\subsubsection{BoxModel}
\par
\indent
\par
\subsubsection{LogicCell}
\par
\indent
\par
\subsubsection{Player}
\par
\indent
\par
\subsubsection{Téléportation}
\par
\indent
\par

\subsection{Classes de la Vue}
\subsubsection{CellDisplay}
\par
\indent
\par
\subsubsection{DisplayBoard}
\par
\indent
\par
\subsubsection{MainWindow}
\par
\indent
\par

\subsection{Classes du Controlleur}
\subsubsection{ControllerBoard}
\par
\indent
\par

\par
\section{Logique du jeu}
\par
\indent
\par
\section{Modèle-Vue-Controlleur}
\par
\indent
\par
\section{Conclusion}
\par
\indent

\par

\end{large}

\end{document}
